\chapter{Conclusion}\label{ch:conclusion}
% Possible improvements by sampling (see \cite{opticsankerst1999optics})
In the present state of huge amounts of data acquisition in various fields,  such as medicine, economy and artificial intelligence, the task of detecting and extracting relevant subspace clusters continues to be highly relevant. However, the current implementations of correlation clustering algorithms only focus on the detection of either local correlation clusters or global correlation cluster and lack the means to find an agglomerated view of both during a single evaluation run. In this thesis, we reviewed existing Correlation Clustering approaches and highlighted their shortcomings concerning their target scope being only applicable for either local or global Correlation Clustering. 

As a solution, we proposed a novel approach for finding both scopes of clustering simultaneously by applying a density-based preprocessing step first and performing a Correlation Clustering algorithm on the resulting locally dense clusters afterwards. This intermediate result contains the local correlation clusters, which represents the local view of the Correlation Clustering. To create the global view, we combined the intermediate clusters by stitching them together if they are similar enough and relabelled all previously disregarded (not dense) points for completeness.

% Our empirical performance analysis in terms of clustering accuracy and runtime was conducted on three different experiments with regards to variable numbers of data objects, amounts of noise and dimensionalities, and additionally compared to the performance measures of its parent algorithm \gls{cash}.
To evaluate the performance in terms of clustering accuracy and runtime, we conducted three different settings for experiments with regards to the number of data objects, amount of noise and dimensionality, and compared those to the performance measures of its parent algorithm \gls{cash}. 
The runtime results yielded that on average, our algorithm, in each runtime setting, performs comparably to \gls{cash} and even gets a slight edge at a comparison in high noise levels. With regards to higher dimensionalities however, we were not able to retrieve representable performance measures due to time and processing power constraints. In terms of clustering performance, our tests, with the best parameter setting previously determined, revealed similar clustering scores as well, with our algorithm having an advantage at higher levels of noise again. However, the tests on the dimensionality also exposed, that both algorithms scale worse for increasing dimensions. At high dimensions, our algorithms best score paled in comparison to \gls{cash} due to the best parameter set being harder to determine since our algorithm requiring more parameters.

All things considered, our first empirical performance analysis yields that, in addition to providing both local and global Correlation Clustering, our algorithm performs equally well compared to original \gls{cash} in terms of the global view, and suggests promising performance in regards to correlation clusterings with arbitrary scopes.

\chapter{Future Work}\label{ch:futurework}
As the topic of correlation clustering with arbitrary scopes is by no means covered and is just at the beginning of the research, we want to provide suggestions for an outlook of potential future directions. 
% To provide an outlook of potential future directions gathered by ideation during the process of the creation of this work, we hope to motivate for future research and present the following contributions.

This work only covered a basic principle of assembling locally dense clusters with global correlation clustering via \gls{dbscan}/\gls{optics} and \gls{cash}. However, these components only served as quick and convenient building blocks to realize an implementation of said principle and by no means have to be the optimal solutions. For future work, we suggest to research and experiment with different, more advanced or modified components, e.g. using modified distance functions in \gls{dbscan}/\gls{optics} to cope for the Curse of Dimensionality, choosing a different Correlation Clustering method for improved correlation results or modifying the assembling method of the local correlation clusters.

As the in-depth evaluation of the impact to the performance measures for every single parameter is very extensive and was not possible in the frame of this thesis, a survey about the different parameters and their impacts with regards to local and global clustering results could be subject to future research.

Another path of future work targets the implementation of our algorithm itself as our work does not satisfy the need for high performance with regards to runtime we recommend to further research and elaborate on strategies to accelerate the execution time of our approach.

