\chapter{Related Work}
\label{sec:Related Work}
This chapter introduces some foundational work on density-based and subspace clustering and also gives an insight into existing approaches to solve the problem of subspace clustering. We also elaborate, where these existing approaches lack in ability and capability and show some of the current optimization approaches.

\section{Clustering}
There are many approaches to gain more information about unknown data. 
(Global) Correlation Clustering, other algorithms so far (ORCLUS \cite{orclusaggarwal2000finding}, LMCLUS \cite{}, 4C, HiCO, ERiC)[1]
Hough Transform [0.5]

The following related work will be elaborated in detail in the chapter \nameref{sec:Foundations}

\section{Hough Transformation}
The Hough Transform originally was introduced in the field of computer vision for edge detection. However it 

\section{CASH}
\ac{CASH}
\cite{CASHachtert2008robust}

\section{DBSCAN}
\ac{DBSCAN}
Density-based Clustering [0.5]

\section{OPTICS}
\cite{opticsankerst1999optics}
% Maybe OPTICS? DIRECTLY COPIED OUT OF \cite{ankerst1999optics}
%  First, almost all clustering algorithms require values for input parameters which are hard to
% determine, especially for real-world data sets containing highdimensional objects. Second, the algorithms are very sensible to
% these parameter values, often producing very different partitionings of the data set even for slightly different parameter settings.
% Third, high-dimensional real-data sets often have a very skewed
% distribution that cannot be revealed by a clustering algorithm using only one global parameter setting. 

\section{Current Optimization Approaches}
    (D-MASC\cite{kazempour2018d, kazempour2019galaxy}, A Galaxy of Correlations etc.) [0.5]
    
