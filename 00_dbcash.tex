\documentclass[pdftex,12pt,a4paper]{report}
\usepackage{dbstmpl}
\usepackage{subfigure}

% Hier die eigenen Daten eintragen
\global\arbeit{Master's Thesis}
\global\titel{Detecting Global Correlated Clusters using Hough Transform through Locally Dense Correlations}
\global\bearbeiter{Long Mathias Yan}
\global\betreuer{Daniyal Kazempour}
\global\aufgabensteller{Dr. Peer Kr"oger}
\global\abgabetermin{14. November 2019}
\global\ort{Munich}
\global\fach{Computer Science}

\begin{document}

% Deckblatt
\deckblatt

% Erklaerung fuer das Pruefungsamt
\erklaerung

% Zusammenfassung
\begin{abstract}
Dieses Dokument dient als Muster f"ur die Ausarbeitung einer \the\arbeit\
an der Lehr- und Forschungseinheit f"ur Datenbanksysteme am Institut f"ur
Informatik der LMU M"unchen.
\end{abstract}

% Inhaltsverzeichnis
\tableofcontents

% Hier beginnt der eigentliche Text
\chapter{Introduction}

In the midst of fast-paced advancements in data gathering we currently are living in the world and age of data abundance. This data however often is present in the form of huge masses of observations consisting of another bunch of features represented as numerical values and thus does not provide much human insight or intuition. Therefore this abundance does not solve our actual problem of information scarcity. Without a given procedure to extract important information out of the data this data abundance would therefore be mostly useless for immediate human processing. One old way to cope with this issue is the manual labeling of each observation by human hand, which however obviously is getting more and more unpopular since with the amount of data this task is already getting inefficient or even infeasible. %To tackle this problem there has been done a lot of research in the fields of Deep- and Unsupervised Learning. 
To tackle this problem there has been lots of different approaches. Either via implicit feature engineering, where we are assuming that the model learns relevant features by itself, or automatic knowledge discovery processes in the field of unsupervised learning, where the goal is to segment datasets by some shared attributes/simplify datasets by aggregating variables with similiarity or detecting anomalies or correlations

In this work we focus on the unsupervised solution for this problem. Namely we 

Clustering in Arbitrary Subspaces based on the Hough transform (CASH) is cool, however its complexity is $\mathcal{O}(n^2)$.
We try to do it better via using a density based approach to prune away irrelevant points and noise..


\chapter{Related Work}
\label{sec:Related Work}
This chapter introduces some foundational work on density-based and subspace/correlation clustering and hopes to give an intuitive insight into existing approaches to solve the problem of subspace clustering. We also elaborate, where these existing approaches lack in ability and capability and show some of the current optimization approaches.

Since sections \nameref{sec:houghintro}, \nameref{sec:cashintro}, \nameref{sec:dbscanintro} and \nameref{sec:OPTICSintro} are essential to our work they will be elaborated in greater detail in the chapter \nameref{sec:Foundations}

\section{Existing Subspace Clustering Algorithms}
There are many approaches to detecting relevant subspaces in data, generally grouped to either finding axis-parallel or arbitrarily oriented subspaces. Our work in particular focuses on the extraction of the second type of subspaces which obviously proves to be a more challenging task since it encompasses the problem of axis-parallel subspace clustering\todor{the first type}. In this section we Like \acrshort{orclus} and \acrshort{4c}, many of them are based on the application of the \gls{pca} to extract the correlations. This section introduces some existing subspace algorithms and explains, in which way those works would or would not be suited for our idea.

\subsection{PCA}

\subsection{ORCLUS}

\subsection{4C and COPAC}

\subsection{HiCO and ERiC}
(Global) Correlation Clustering, other algorithms so far (ORCLUS \cite{orclusaggarwal2000finding}, LMCLUS \cite{}, 4C, HiCO, ERiC)[1]

Since many of the existing correlation clustering algorithms rely on \gls{PCA} they also come with its limitations.

\section{Hough Transformation}\label{sec:houghintro}
The Hough Transform originally was introduced by \textcite{houghOriginal1962method} and extended by \textcite{rosenfeld1969picture} in the field of computer vision for edge detection\cite{houghhistoryhart2009hough}. The initial purpose of the Hough transform was a technique to detect colinear points in an image space but has since then found various other applications in fields like image processing/analysis~\cite{rosenfeld1969picture,ballard1981generalizing}, computer vision~\cite{davies2004machine} and subspace clustering\cite{CASHachtert2008robust}.
The basic idea of the Hough transform is the transformation of all points $p_i = (x_i,y_i)$ in a 2-dimensional image space $\mathcal{D} \subseteq \R^2$ to functions $f_{p_i}$ in a 2-dimensional parameter space $\mathcal{P} \subseteq \R^2$, also known as Hough space\cite{illingworth1988survey}. This is can be done by e.g. taking a representation of a point $p$ as all of its concurrent lines $y_p = m \cdot x_p + t$ and rearranging it to $m_{p} = - \frac{1}{x_p} \cdot t_{p} + \frac{y_p}{x_p}$ which produces a straight with slope $m$ and y-intersect $t$ in a $(m,t)$-parameter space. Since each point in parameter space represents a particular $(m,t)$-setting, multiple functions close to each other implies that their respective points have similar $(m,t)$-settings as well. The correlation clustering objective therefore transforms to a density-based clustering objective in parameter space, with the added benefit of being able to detect correlating points regardless of their distance to each other in data space. This property is exploited by e.g. evaluating the whole parameter space in a grid with a voting scheme or by smartly splitting the parameter space in \autoref{sec:houghintro} to detect linear correlations. 
\todor{Ich plagiere mich selbst. 1zu1 aus unterem abschnitt}
% \begin{figure}
%     \centering
%     \includegraphics{figures/HoughMXT.pdf}
%     \caption{Caption}
%     \label{fig:houghmxt}
% \end{figure}\todor{eher keine bilder in related work}

\section{CASH}\label{sec:cashintro}
The global correlation clustering algorithm \gls{cash} extends the use case of the Hough Transformation to the detection of arbitrary-dimensional subspaces by augmenting the initial rigid 2-dimensional definition to a multi-dimensional one. Furthermore \gls{cash} introduces a improved search strategy for detecting regions of high intersections in parameter space to improve the efficiency compared to the basic grid search \cite{CASHachtert2008global}. Instead of doing an extensive count operation/accumulation of intersections over a fixed interval, \gls{cash} successively splits the whole parameter space by its axes and only further evaluates the split hypercuboids if they contain enough intersections. This is repeated until a certain count of splits is reached and only then hypercuboids with enough intersections are considered as linear correlations. Since our work focuses on ``\textit{Detecting Global Correlated Clusters using Hough Transform through Locally Dense Correlations}'', we use \gls{cash} to cope with multi-dimensional data and detect our locally dense correlations. Additionally we will use \gls{cash} as a performance baseline for our descendant algorithm to compare them in a global setting.
A more profound explanation to the transformation and its usage in \gls{cash} will be given in \autoref{ssec:houghindepth}.


\section{DBSCAN}\label{sec:dbscanintro}
\citeauthor{DBSCANEKSX96} created a foundational algorithm with \gls{dbscan}. With over 16000 citations on google scholar as of December 2019 it is one of the most influential works created in the field of density-based clustering and a basis to many clustering approaches, not only restricted to density-based clustering. As its name reveals it is an algorithm which detects points in dense vicinity and groups them together. For a measure of density \gls{dbscan} utilizes two parameters. One to specify the minimum amount of neighboring points in a close vicinity and one to specify the range/radius of that vicinity. Points fullfilling these conditions are called \textit{core points} and represent the dense \textit{core} of a cluster. The border of a dense cluster is composed of \textit{border points}. They are points which themselves do not possess a dense neighborhood but are still in the vicinity of core points. In contrast to $k$-means-like partitioning clustering\cite{kmeansmacqueen1967some}, \gls{dbscan} is able to find not only non-convex shapes, but also any arbitrary shape of a particular density. Since these arbitrary shaped clusters preserves their correlations and our goal is the assembly of locally dense correlations to global correlations, we expect to obtain good results by partitioning our data space via a density-based algorithm.

\section{OPTICS}\label{sec:OPTICSintro}
A disadvantage of \gls{dbscan} is its dependence on a fixed global parameter setting defining the \textit{minimal} detectable density. Assuming a global linear correlation to have low fluctuations in variance and different global linear correlations having various other variances\todor{can i assume this? i have to do some assumptions right?}, finding clusters with single densities would be more advantageous to our algorithm. We therefore adopted the use of \gls{optics} instead, an improvement/extension of \gls{dbscan}, which enables us to extract single densities more accurately \cite{opticsankerst1999optics}. Since \gls{dbscan} and \gls{optics} are the foundations of the partitioning step we will give a more comprehensive explanations to those two algorithms as well (c.f. \autoref{ssec:DBSCANindepth} and \autoref{ssec:OPTICSindepth}).
% Maybe OPTICS? DIRECTLY COPIED OUT OF \cite{ankerst1999optics}
%  First, almost all clustering algorithms require values for input parameters which are hard to
% determine, especially for real-world data sets containing highdimensional objects. Second, the algorithms are very sensible to
% these parameter values, often producing very different partitionings of the data set even for slightly different parameter settings.
% Third, high-dimensional real-data sets often have a very skewed
% distribution that cannot be revealed by a clustering algorithm using only one global parameter setting. 

\section{Current Optimization Approaches}
(D-MASC\cite{kazempour2018d, kazempour2019galaxy}, A Galaxy of Correlations etc.) [0.5] \todor{DMASC ist eigtl nur related work zu CASH aber relevant fuer uns?}
    

\chapter{Methods}
\section{Mathematical and Algorithmic Foundations [4]}
\subsection{Hough Transformation [2]}
\subsection{CASH [1]}
\subsection{DBSCAN [1]}

\section{The Algorithm [5]}

\subsection{Finding Dense Clusters}
Find dense clusters using DBSCAN → elaborate on bounding boxes around dense areas [1]

\subsection{Finding Linear Correlations}
Apply CASH onto dense Clusters [2]

\subsection{Stitching}

Assembly of local linear correlations [2]
\chapter{Evaluation}
To measure the performance of our algorithm, we evaluated \todor{werden tests evaluiert, oder algorithmen?} our algorithm on custom data sets, generated via the method mentioned in \autoref{sec:datagen}. Each of the data sets linear correlations are uniquely labeled and subsequently added noise points relabeled to a nearby linear correlation if the point was within a certain vicinity from the correlation away. We defined the threshold of the vicinity as $jitter_{thresh}$ which merges the noise points if the euclidean distance to a nearby linear correlation $distance_{point\rightarrow hyperplane} = \frac{\Vec{n}\cdot\Vec{x}+\delta}{|\Vec{n}|}$ is smaller than the threshold. These data sets are served as our \textit{groundtruth}. Based on these groundtruths we compared our local-global combining clustering approach's results with its ancestor \gls{cash}'s results in the accuracy of their labeling based on the \gls{ari} and \gls{nmi} score and its runtime performance. 

\section{Setup}


All following tests were executed in docker containers running on a Virtual Machine with an x86\_64 bit architecture, 48 CPUs and 246GB RAM. 

(which data sets have been used? How many data objects? How many clusters? Which programming language and libraries? On which hardware?) [0.5]

\section{Parameters available and their impacts}
Our algorithm depends on many subprocesses which come with several parameters themselves. In this section we discuss their meaning and especially their impact for the clustering results.

\subsection{Metrics: CosineSimiliarity(n1,n2), CosineSimiliarity(n1,n2) + EuclidianDistance(d) [2-3]}

\subsection{Median vs.  Mean [2-3]}

\section{Results between Dense approach with stitching and Global approach (runtime included) [2-3]
}

\section{Test on real world data set(s) [1]}

%evtl. "Hyperparameter sensitivity" d.h. wie 'empfindlich' ist das verfahren bzgl. welchen Parameter Einstellungen?
\chapter{Conclusion}\label{ch:conclusion}
% Possible improvements by sampling (see \cite{opticsankerst1999optics})
In the present state of huge amounts of data acquisition in various fields,  such as medicine, economy and artificial intelligence, the task of detecting and extracting relevant subspace clusters continues to be highly relevant. However, the current implementations of correlation clustering algorithms only focus on the detection of either local correlation clusters or global correlation cluster and lack the means to find an agglomerated view of both during a single evaluation run. In this thesis, we reviewed existing Correlation Clustering approaches and highlighted their shortcomings concerning their target scope being only applicable for either local or global Correlation Clustering. 

As a solution, we proposed a novel approach for finding both scopes of clustering simultaneously by applying a density-based preprocessing step first and performing a Correlation Clustering algorithm on the resulting locally dense clusters afterwards. This intermediate result contains the local correlation clusters, which represents the local view of the Correlation Clustering. To create the global view, we combined the intermediate clusters by stitching them together if they are similar enough and relabelled all previously disregarded (not dense) points for completeness.

% Our empirical performance analysis in terms of clustering accuracy and runtime was conducted on three different experiments with regards to variable numbers of data objects, amounts of noise and dimensionalities, and additionally compared to the performance measures of its parent algorithm \gls{cash}.
To evaluate the performance in terms of clustering accuracy and runtime, we conducted three different settings for experiments with regards to the number of data objects, amount of noise and dimensionality, and compared those to the performance measures of its parent algorithm \gls{cash}. 
The runtime results yielded that on average, our algorithm, in each runtime setting, performs comparably to \gls{cash} and even gets a slight edge at a comparison in high noise levels. With regards to higher dimensionalities however, we were not able to retrieve representable performance measures due to time and processing power constraints. In terms of clustering performance, our tests, with the best parameter setting previously determined, revealed similar clustering scores as well, with our algorithm having an advantage at higher levels of noise again. However, the tests on the dimensionality also exposed, that both algorithms scale worse for increasing dimensions. At high dimensions, our algorithms best score paled in comparison to \gls{cash} due to the best parameter set being harder to determine since our algorithm requiring more parameters.

All things considered, our first empirical performance analysis yields that, in addition to providing both local and global Correlation Clustering, our algorithm performs equally well compared to original \gls{cash} in terms of the global view, and suggests promising performance in regards to correlation clusterings with arbitrary scopes.

\chapter{Future Work}\label{ch:futurework}
As the topic of correlation clustering with arbitrary scopes is by no means covered and is just at the beginning of the research, we want to provide suggestions for an outlook of potential future directions. 
% To provide an outlook of potential future directions gathered by ideation during the process of the creation of this work, we hope to motivate for future research and present the following contributions.

This work only covered a basic principle of assembling locally dense clusters with global correlation clustering via \gls{dbscan}/\gls{optics} and \gls{cash}. However, these components only served as quick and convenient building blocks to realize an implementation of said principle and by no means have to be the optimal solutions. For future work, we suggest to research and experiment with different, more advanced or modified components, e.g. using modified distance functions in \gls{dbscan}/\gls{optics} to cope for the Curse of Dimensionality, choosing a different Correlation Clustering method for improved correlation results or modifying the assembling method of the local correlation clusters.

As the in-depth evaluation of the impact to the performance measures for every single parameter is very extensive and was not possible in the frame of this thesis, a survey about the different parameters and their impacts with regards to local and global clustering results could be subject to future research.

Another path of future work targets the implementation of our algorithm itself as our work does not satisfy the need for high performance with regards to runtime we recommend to further research and elaborate on strategies to accelerate the execution time of our approach.



\appendix
\chapter{Results}

Results with Figures?

% Abbildungsverzeichnis (kann auch nach dem Inhaltsverzeichnis kommen)
\listoffigures

% Tabellenverzeichnis (kann auch nach dem Inhaltsverzeichnis kommen)
\listoftables

% Literaturverzeichnis
\bibliographystyle{dbstmpl}    % verwendet dbstmpl.bst
% alternative, vorinstallierte Stile sind z.B. plain oder abbrv
\bibliography{dbstmpl}         % verwendet dbstmpl.bib

\end{document}
